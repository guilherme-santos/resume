% LaTeX file for resume
% This file uses the resume document class (res.cls)

\documentclass[margin]{res}
\usepackage [brazil]{babel}    % nomes e hifenaçã em português

\usepackage{t1enc}             % Permite digitar os acentos de forma normal
\usepackage[utf8]{inputenc}

\topmargin=-0.5in              % start text higher on the page
\setlength{\textheight}{10in}  % increase text height to fit resume on 1 page

\begin{document}

\name{\textit{Guilherme Silveira dos Santos}}
\address{Berlin, DE \\ xguiga@gmail.com \\ Phone: +49 (173) 979-9383 }

\begin{resume}

\section{Summary}
I'm a Computer Engineer since 2011 but I started programming long time before that. I consider myself autodidact because I like to learn new technologies all the time and I can do it very quickly. I have started to study PHP at home since I was teenager what give me good knowledge to develop many web applications. During the college and internships I learned other programming languages, such as C, C++, Java and Lua and I could improve myself with other skills.

I've been working for long time as a freelancer developer and most of the time involving web development, as a result I became an experienced web developer with big expertise in PHP, JavaScript, CSS and HTML. Apart from that I improved myself as DevOps creating CLI and scripts to automate deploys, configure servers, etc. In 2015 I started working with Node.js and Go.

I worked in companies with several different technologies and it made me an open minded developer, with can-do attitude and a good interpersonal relationship. Currently I'm working mainly with Go and PHP as Backend Developer in Berlin, but looking for new challenges.

\section{Education} UNIVALI University, BSc in Computer Engineering, June 2011.

\section{Experience}
\vspace{-0.1in}
    \begin{tabbing}
    \hspace{2.3in}\= \hspace{1.7in}\= \kill % set up two tab positions
    \textbf{Foodora / Delivery Hero}    \>\>\textbf{Jan 2018 - Present}\\
    \textit{Senior Backend Developer}\\
    \textbf{Main Technologies}: Go, PHP, Symfony, ZendFramework, MySQL, ElasticSearch, \\Redis, Docker, AWS, Kubernetes;
    \end{tabbing}\vspace{-20pt}      % suppress blank line after tabbing
    \vspace{2mm}
I'm working on Pandora (Foodora + Foodpanda) platform, moving from Monolithic application to micro-services. Also I'm developing new libraries and helping to define new guidelines and best practices to Go projects to level the whole Pandora team.

\section{Experience}
\vspace{-0.1in}
    \begin{tabbing}
    \hspace{2.3in}\= \hspace{1.7in}\= \kill % set up two tab positions
    \textbf{Ridelink}    \>\>\textbf{Sep 2016 - Dec 2017}\\
    \textit{Technical Lead}\\
    \textbf{Main Technologies}: PHP, Symfony, Silex, Go, Python, MySQL, ElasticSearch, \\Redis, Docker, AWS;
    \end{tabbing}\vspace{-20pt}      % suppress blank line after tabbing
    \vspace{2mm}
I was a member of a polyglot team responsible for developing and improving services aimed to build a car sharing platform. Most of the micro-services were developed using PHP with frameworks Symfony and Silex; however, there were also projects developed in Go and Python. All services were making use of AWS platform, such as ECS, SQS, ElasticCache, ElasticSearch, RDS and CloudWatch. In addition, I have developed tools to make as easy as possible the continuous deployment. Testing was part of the company’s culture, hands on PHPUnit, Prophesize and Behat for PHP projects, and for Go standard library was enough.

\vspace{-0.1in}
    \begin{tabbing}
    \hspace{2.3in}\= \hspace{1.7in}\= \kill % set up two tab positions
    \textbf{Neoway Business Solution}    \>\>\textbf{Jun 2015 - Sep 2016}\\
    \textit{Senior Backend Developer}\\
    \textbf{Main Technologies}: Go, ElasticSearch, MongoDB, RabbitMQ, Docker, Rkt,\\CoreOS, AWS;
    \end{tabbing}\vspace{-20pt}      % suppress blank line after tabbing
    \vspace{2mm}
Development of Big Data platform to Market Intelligence. We were using mainly Go but Node.js was also used in a couple of projects. I developed several RESTful API to communicate with different backends like: ElasticSearch, MongoDB and RabbitMQ which gave me some experience on how to use and configure them.

We loved DevOps culture here, for that I've developed some tools to automate our deploy at AWS. We often wrote unit and integration tests to make our deploy as continuous as possible using GitLab flow and Docker/Rkt containers.

\newpage

\vspace{-0.1in}
    \begin{tabbing}
    \hspace{2.3in}\= \hspace{1.7in}\= \kill
    \textbf{uTech Tecnologia}    \>\>\textbf{Nov 2014 - Feb 2015}\\
    \textit{Full Stack Developer}\\
    \textbf{Main Technologies}: C++, Qt, QML, JavaScript, SIP;
    \end{tabbing}\vspace{-20pt}
    \vspace{2mm}
Outsourced development of a Softphone to integrate with company’s platform. The biggest challenge was be multi-platform, running in Windows 7, Linux and Mac OS X. The software was developed using C++, QML and JavaScript through Qt library and PJSIP as SIP stack.

\vspace{-0.1in}
    \begin{tabbing}
    \hspace{2.3in}\= \hspace{1.7in}\= \kill
    \textbf{GIOX Tecnologia}    \>\>\textbf{Mar 2013 - Present}\\
    \textit{CEO and Full Stack Developer}\\
    \textbf{Main Technologies}: PHP, Go, JavaScript, HTML5, CSS3, MySQL, MongoDB,\\Docker, NSQ, Phing, Ansible, CoreOS, SOAP, XML;
    \end{tabbing}\vspace{-20pt}
    \vspace{2mm}
As a entrepreneur I founded GIOX in 2013, a company focused in creating SAS ERP for small businesses with electronic invoice (NF-e). To develop the project I have used mainly PHP as backend language with ZendFramework 2, Silex, Doctrine2 and PHPUnit. MySQL was my choice for database and NSQ as message broker. I also have been writing microservices in Go with MongoDB. For the frontend I have used jQuery, Backbone.js, Underscore.js and Bootstrap3 frameworks.

I deployed and built the whole infrastructure for GIOX ERP, as a result I improved myself as SysAdmin and DevOps. Now, the project is running on Digital Ocean and I make use of Ansible, Git and Phing to automate tasks and maintain Linux servers.

\vspace{-0.1in}
    \begin{tabbing}
    \hspace{2.3in}\= \hspace{1.7in}\= \kill
    \textbf{Digitro Tecnologia}    \>\>\textbf{Dec 2009 - Mar 2013}\\
    \textit{Backend Developer}\\
    \textbf{Main Technologies}: C, C++, Lua, SIP, uCLinux, GStreamer, ShellScript, Blackfin;
    \end{tabbing}\vspace{-20pt}
    \vspace{2mm}
Embedded development of an IP Phone touch screen with color display using Blackfin processor with uCLinux distribution. We used u-boot, EFL graphic library, GLib, GObject, GDBus, CppUTest and Sofia-SIP as SIP stack.

Speaker recognition: Web service responsible for creating audio models from the voice and storing it. The voice models were used later to identify someone talking on a audio recording. Tools used: Lua, lighthttpd, GStreamer, fastcgi, MongoDB.

Keyword spotting: Middleware used to perform text search on audio using a proprietary protocol to communicate with clients. Tools used: Lua, C, GStreamer.

Flash Audio streaming server: Web service responsible for loading audio recordings in different audio codecs, transcode then, apply filters and effects and send to the client through the RTMP protocol. Tools used: C++, C, GStreamer, Monit.

\section{Skills Base} \textit{Programming Languages}: Go, PHP, Python, JavaScript, C, C++, Lua and Java;

	\textit{Databases}: MongoDB, MySQL, Redis, LevelDB and a little of PostgreSQL;

	\textit{Frameworks}: Cobra, Silex, Symfony, ZendFramework2/3, Docktrine2, jQuery, Backbone.js, Underscore.js ;

	\textit{Agile practices}: TDD, SCRUM, Kanban, Pair programming, Clean code, Code Review;

	\textit{Tools}: Git, Makefile, Docker-Compose, Phing, Bower, Ansible, Jenkins, GitLab CI;

	\textit{Languages}: Fluent in Portuguese, Intermediate in English;

	\textit{Others}: ElasticSearch, RabbitMQ, NSQ, Docker, Rkt, AWS, Kubernetes;

\section{More Info} \textit{Linkedin}: https://www.linkedin.com/in/guilhermesilveirasantos

    \textit{Github}: https://github.com/guilherme-santos

\end{resume}
\end{document}
