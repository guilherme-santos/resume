%% start of file `template.tex'.
%% Copyright 2006-2013 Xavier Danaux (xdanaux@gmail.com).
%
% This work may be distributed and/or modified under the
% conditions of the LaTeX Project Public License version 1.3c,
% available at http://www.latex-project.org/lppl/.


\documentclass[11pt,a4paper,sans]{moderncv}        % possible options include font size ('10pt', '11pt' and '12pt'), paper size ('a4paper', 'letterpaper', 'a5paper', 'legalpaper', 'executivepaper' and 'landscape') and font family ('sans' and 'roman')

% moderncv themes
\moderncvstyle{banking}                            % style options are 'casual' (default), 'classic', 'oldstyle' and 'banking'
\moderncvcolor{red}                                % color options 'blue' (default), 'orange', 'green', 'red', 'purple', 'grey' and 'black'
%\renewcommand{\familydefault}{\sfdefault}         % to set the default font; use '\sfdefault' for the default sans serif font, '\rmdefault' for the default roman one, or any tex font name
%\nopagenumbers{}                                  % uncomment to suppress automatic page numbering for CVs longer than one page

% character encoding
\usepackage[utf8]{inputenc}                       % if you are not using xelatex ou lualatex, replace by the encoding you are using
%\usepackage{CJKutf8}                              % if you need to use CJK to typeset your resume in Chinese, Japanese or Korean

% adjust the page margins
\usepackage[scale=0.75]{geometry}
%\setlength{\hintscolumnwidth}{3cm}                % if you want to change the width of the column with the dates
%\setlength{\makecvtitlenamewidth}{10cm}           % for the 'classic' style, if you want to force the width allocated to your name and avoid line breaks. be careful though, the length is normally calculated to avoid any overlap with your personal info; use this at your own typographical risks...

% personal data
\name{Guilherme}{Santos}
\title{Resumé title}                               % optional, remove / comment the line if not wanted
\address{Oscar João Costa, 173}{Florianopolis}{Brazil}% optional, remove / comment the line if not wanted; the "postcode city" and and "country" arguments can be omitted or provided empty
\phone[mobile]{+55~(48)~9640~3883}                   % optional, remove / comment the line if not wanted
%\phone[fixed]{+2~(345)~678~901}                    % optional, remove / comment the line if not wanted
%\phone[fax]{+3~(456)~789~012}                      % optional, remove / comment the line if not wanted
\email{xguiga@gmail.com}                               % optional, remove / comment the line if not wanted
\homepage{linkedin.com/in/guilhermesilveirasantos}                         % optional, remove / comment the line if not wanted
\extrainfo{github.com/guilherme-santos}                 % optional, remove / comment the line if not wanted
\photo[64pt][0.4pt]{picture}                       % optional, remove / comment the line if not wanted; '64pt' is the height the picture must be resized to, 0.4pt is the thickness of the frame around it (put it to 0pt for no frame) and 'picture' is the name of the picture file
\quote{Some quote}                                 % optional, remove / comment the line if not wanted

% to show numerical labels in the bibliography (default is to show no labels); only useful if you make citations in your resume
%\makeatletter
%\renewcommand*{\bibliographyitemlabel}{\@biblabel{\arabic{enumiv}}}
%\makeatother
%\renewcommand*{\bibliographyitemlabel}{[\arabic{enumiv}]}% CONSIDER REPLACING THE ABOVE BY THIS

% bibliography with mutiple entries
%\usepackage{multibib}
%\newcites{book,misc}{{Books},{Others}}
%----------------------------------------------------------------------------------
%            content
%----------------------------------------------------------------------------------
\begin{document}
%-----       letter       ---------------------------------------------------------
% recipient data
\recipient{}{}
\date{June 25, 2016}
\opening{Hi,}
\closing{Yours faithfully,}
\enclosure[Attached]{curriculum vit\ae{}}          % use an optional argument to use a string other than "Enclosure", or redefine \enclname
\maketitle

I’m a Computer Engineer since 2011, but I started programming long time before that. I consider myself autodidact because I like to learn new technologies all the time and I can do it very quickly. I have started to study PHP at home since I was teenager, what give me good knowledge to develop many web applications. During the college and internships I learned other programming languages, such as C, C++, Java and Lua and I could improve myself with other skills.

I’ve been working for long time as a freelancer developer and most of the time involving web development, as a result I became an experienced web developer with big expertise in PHP, JavaScript, CSS and HTML. In the last year I started working with Node.js and Golang.

I’ve already contributed in many open source projects like Enlightenment Foundation Libraries (EFL), Zend Framework and other small JavaScript and Golang libraries. I like to work with open source projects and I do it in my spare time.

I worked in companies with several different technologies and it made me an open-minded developer, with can-do attitude and a good interpersonal relationship. Currently I’m working with Golang as Full Stack Developer in south Brazil, but looking for new challenges around the world.

\vspace{8mm}

\makeletterclosing



\end{document}


%% end of file `template.tex'.
